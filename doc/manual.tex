\documentclass[a4paper]{scrartcl}
\usepackage[T1]{fontenc}			
\usepackage[utf8]{inputenc}	
\usepackage{lmodern}			
\usepackage{microtype}	
\usepackage{hyperref}

\title{Orthograph}
\author{Malte Petersen}
\date{\today}

\begin{document}
\maketitle
\tableofcontents

\section{Notes on HMMer3}

HMMer3 uses three notions to report endpoints of the alignment regions: Two
give the endpoints of the reported local alignment with respect to the query
model (``hmm from'' and ``hmm to'') and the target sequence (``ali from'' and
``ali to''). The third defines the \emph{envelope} (``env from'' and ``env to'')
of the domain's location on the target sequence. It is mostly a little (or a
lot) wider than than what HMMer thinks is a reasonably confident alignment and
represents a subsequence whose endpoints are only fuzzily inferrable.

Orthograph uses the envelope coordinates for its analysis, and this is normally
fine when working with high-scoring domains that are not close to each other.
However, be aware that the envelope coordinates of hmmsearch hits can and will
overlap when two weaker-scoring domains are close to each other, and this may
lead to some hits getting excluded because overlaps are not allowed in ortholog
regions. 

\section{SHA-256 checksums}

Orthograph uses SHA-256 checksums of the sequences to unambiguously identify
them in the database. Please note that while SHA-256 has a tremendously large
hash space and has not yet been compromised, i.e., there is no known collision
attack against the algorithm, a collision (two different objects producing the
same checksum) is still possible. However, the probability for that event is
vanishingly small: There need to be more than $10^{15}$ objects for the SHA-256
hashes to exceed a collision probability of $10^{-18}$. Since the hash space is
expected to contain only a number of sequence objects in the range of $10^6$ to
$10^{12}$, it is statistically safe to assume that every checksum is unique. 

Thomas Pornin said on \href{http://stackoverflow.com/questions/4014090/is-it-safe-to-ignore-the-possibility-of-sha-collisions-in-practice}{http://stackoverflow.com/questions/4014090/is-it-safe-to-ignore-the-possibility-of-sha-collisions-in-practice}:

\begin{quote}
The usual answer goes thus: what is the probability that a rogue asteroid
crashes on Earth within the next second, obliterating
civilization-as-we-know-it, and killing off a few billion people? It can be
argued that any unlucky event with a probability lower than that is not
actually very important.

If we have a ``perfect'' hash function with output size $n$, and we have $p$
messages to hash (individual message length is not important), then the
probability of a collision is about $\frac{p^2}{2^{n+1}}$ (this is an
approximation which is valid for ``small'' $p$, i.e., substantially smaller
than $2^{\frac{n}{2}}$). For instance, with SHA-256 ($n=256$) and one billion
messages ($p=10^9$) the probability is about $4.3 \cdot 10^{-60}$.

A mass-murderer space rock happens about once every 30 million years on
average. This leads to a probability of such an event occurring in the next
second to about $10^{-15}$. That's \emph{45 orders of magnitude} more probable
than the SHA-256 collision. Briefly stated, if you find SHA-256 collisions
scary then your priorities are wrong.
\end{quote}

\end{document}
